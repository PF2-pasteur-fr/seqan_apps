\nonstopmode{}
\documentclass[a4paper]{book}
\usepackage[times,inconsolata,hyper]{Rd}
\usepackage{makeidx}
\usepackage[utf8,latin1]{inputenc}
% \usepackage{graphicx} % @USE GRAPHICX@
\makeindex{}
\begin{document}
\chapter*{}
\begin{center}
{\textbf{\huge Package `ngsroi'}}
\par\bigskip{\large \today}
\end{center}
\begin{description}
\raggedright{}
\item[Type]\AsIs{Package}
\item[Title]\AsIs{NGS Regions of Interest Analysis}
\item[Version]\AsIs{1.0}
\item[Date]\AsIs{2013-04-12}
\item[Maintainer]\AsIs{Manuel Holtgrewe }\email{manuel.holtgrewe@fu-berlin.de}\AsIs{}
\item[Description]\AsIs{Routines for I/O of NGS ROI files and manipulation thereof.}
\item[License]\AsIs{BSD + file LICENSE}
\item[URL]\AsIs{}\url{http://www.seqan.de/projects/ngs\_roi}\AsIs{}
\item[Author]\AsIs{Bernd Jagla Developer [aut, cre, cph],
Manuel Holtgrewe Developer [ctb]}
\end{description}
\Rdcontents{\R{} topics documented:}
\inputencoding{utf8}
\HeaderA{ngsroi-package}{NGS Regions of Interest Analysis}{ngsroi.Rdash.package}
\aliasA{ngsroi}{ngsroi-package}{ngsroi}
\keyword{package}{ngsroi-package}
%
\begin{Description}\relax
Routines for I/O of NGS ROI files and manipulation thereof.
\end{Description}
%
\begin{Details}\relax

\Tabular{ll}{
Package: & ngsroi\\{}
Type: & Package\\{}
Version: & 0.1\\{}
Date: & 2013-04-12\\{}
License: & BSD\_3\_clause\\{}
}
\end{Details}
%
\begin{Author}\relax
Bernd Jagla <bernd.jagla@pasteur.fr>\\{}
Manuel Holtgrewe <manuel.holtgrewe@fu-berlin.de>

Maintainer: Manuel Holtgrewe <manuel.holtgrewe@fu-berlin.de>
\end{Author}
%
\begin{References}\relax
Jagla, B, Holtgrewe, M, Reinert, K: NGS ROI. To appear.
\end{References}
%
\begin{Examples}
\begin{ExampleCode}
library(ngsroi)

# Load ROI file into data.frame.
#roi = readROI("dmel.bowtie.sam.roi")

# Compute some metrics.
#roi$min = as.numeric(lapply(roi$counts ,min))
#roi$median =  as.numeric(lapply(roi$counts ,median))
#roi$mean =  as.numeric(lapply(roi$counts ,mean))
#roi$quantile75 =  as.numeric(lapply(roi$counts ,quantile, probs=0.75))
#roi$quantile95 =  as.numeric(lapply(roi$counts ,quantile, probs=0.95))

# Write data.frame into ROI file again.
#writeROI(roi, "dmel.bowtie.sam.trans.roi");
\end{ExampleCode}
\end{Examples}
\inputencoding{utf8}
\HeaderA{readROI}{Read ROI file.}{readROI}
\keyword{\textbackslash{}textasciitilde{}kwd1}{readROI}
\keyword{\textbackslash{}textasciitilde{}kwd2}{readROI}
%
\begin{Description}\relax
Read ROI file into data.frame.
\end{Description}
%
\begin{Usage}
\begin{verbatim}
readROI(file.name)
\end{verbatim}
\end{Usage}
%
\begin{Arguments}
\begin{ldescription}
\item[\code{file.name}] 
The path to the ROI file to read.

\end{ldescription}
\end{Arguments}
%
\begin{Value}
The function returns a data.frame with the data from the ROI file.
\end{Value}
%
\begin{Examples}
\begin{ExampleCode}
##---- Should be DIRECTLY executable !! ----
##-- ==>  Define data, use random,
##--	or do  help(data=index)  for the standard data sets.

## The function is currently defined as
function (fpname) 
{
    fi <- function(x, i) {
        x[i]
    }
    fni <- function(x, i) {
        as.numeric(x[[i]])
    }
    getVec <- function(x) {
        mylen = as.numeric(x[5])
        veclen = length(x)
        as.numeric(unlist(strsplit(x[veclen], ",")))
    }
    getVals <- function(y, x, columnNames) {
        veclen = length(columnNames)
        t1 = veclen - 1
        for (colN in c(7:t1)) {
            y[, columnNames[colN]] = unlist(lapply(x, fni, colN))
        }
        return(y)
    }
    con = gzfile(fpname)
    rLines = readLines(con)
    close(con)
    values = rLines[substr(rLines, 1, 1) != "#"]
    values = strsplit(values, "\t")
    if (length(rLines[substr(rLines, 1, 2) == "##"]) == 0) {
        columnNames = unlist(list("##ref", "begin_pos", "end_pos", 
            "region_name", "length", "strand", "max_count", "cg_content", 
            "counts"))
    }
    else {
        columnNames = unlist(strsplit(rLines[substr(rLines, 1, 
            2) == "##"], "\t"))
    }
    df = data.frame(ref = unlist(lapply(values, fi, 1)), begin_pos = as.integer(unlist(lapply(values, 
        fni, 2))), end_pos = as.integer(unlist(lapply(values, 
        fni, 3))), region_name = unlist(lapply(values, fi, 4)), 
        length = as.integer(unlist(lapply(values, fni, 5))), 
        strand = unlist(lapply(values, fi, 6)))
    df = getVals(df, values, columnNames)
    df$counts = lapply(values, getVec)
    df$counts = lapply(df$counts, unlist)
    roiNames = names(df)
    return(df)
  }
\end{ExampleCode}
\end{Examples}
\inputencoding{utf8}
\HeaderA{writeROI}{Write ROI file.}{writeROI}
\keyword{\textbackslash{}textasciitilde{}kwd1}{writeROI}
\keyword{\textbackslash{}textasciitilde{}kwd2}{writeROI}
%
\begin{Description}\relax
Write data.frame to a ROI file.
\end{Description}
%
\begin{Usage}
\begin{verbatim}
writeROI(roi, file.name)
\end{verbatim}
\end{Usage}
%
\begin{Arguments}
\begin{ldescription}
\item[\code{roi}] 
The data.frame to write to file.

\item[\code{file.name}] 
The path to the file to write to.

\end{ldescription}
\end{Arguments}
%
\begin{Examples}
\begin{ExampleCode}
##---- Should be DIRECTLY executable !! ----
##-- ==>  Define data, use random,
##--	or do  help(data=index)  for the standard data sets.

## The function is currently defined as
function (roi, fpname) 
{
    fpConn <- file(fpname, "w")
    colCount = 1
    write("# ROI written from R", fpConn, append = F)
    colNames = names(roi)
    colNames = colNames[!colNames == "counts"]
    outStr = ""
    outStr = paste(c(outStr, "##", colNames[1]), collapse = "")
    for (colIds in c(2:length(colNames))) {
        outStr = paste(c(outStr, colNames[colIds]), collapse = "\t")
    }
    outStr = paste(c(outStr, "counts"), collapse = "\t")
    write(outStr, file = fpConn, append = TRUE)
    apply(roi, 1, writeRoiLine, fpConn)
    close(fpConn)
  }
\end{ExampleCode}
\end{Examples}
\printindex{}
\end{document}
